\documentclass[sigconf,nonacm]{acmart}

\title{Code of Conduct}
\author{Eva Miesen, Lukas Milieška, Rens Pols, Sem van der Weijden, Wing Wong, Xiaoyu Du}

% remove overhead that is used in regular ACM papers
\settopmatter{printacmref=false} % box after abstract
\renewcommand\footnotetextcopyrightpermission[1]{} % copyright on first page
\pagestyle{plain} % running headers

\usepackage[utf8]{inputenc}
\usepackage[T1]{fontenc}

\begin{document}

\maketitle

\tableofcontents


\section{Shared team values}
The following points outline the shared values of Group 25:

\begin{enumerate}
    \item \textbf{Respect}: Members should refrain from insulting or demeaning others based on their opinions, appearance, race, or personality.
    \item \textbf{Integrity}: Members should not engage in humiliating, ridiculing or harming another team member.
    \item \textbf{Openness}: Every opinion is valued and should be discussed without being forced.
    \item \textbf{Inclusivity}: All members, regardless of their natural vocalness, should have opportunities to express their opinions and ideas.
    \item \textbf{Casual Professionalism}: Members should be professional while maintaining a relaxed, casual and enjoyable work atmosphere.
    \item \textbf{Leadership by Example}: Each member leads by example: Be accountable, Treat others with respect, and Demonstrate a passion for our project.
    \item \textbf{Open Communication}: Members should prioritise open and honest communication with other members to ensure transparency and trust.
    \item \textbf{Team Effort}: Members of the team are aware that the project is a group effort and their actions will also reflect their team members.
    \item \textbf{Commitment}: Regardless of skill or prior experience all members should make their best effort in contributing and committing to the project.
    \item \textbf{Resource Sharing:} Members should share resources, such as knowledge, skills, and tools, to leverage each member's strengths and achieve collective results.
\end{enumerate}


\section{Assignment Description}
The assignment is a way for each team member to simulate working in a real development environment. However, a big difference is instead of having set roles per person, like in the ‘real world’, every member gets to experiment with every aspect. The general aim is to make sure every member of the team gets to work on or learn about every little part of the project. By doing this, the application shall be made by the whole team with every member being invested in the project as a whole.

A basic description of the assignment is an “expense calculator”. The project itself entails us making an application where the users can keep track of and edit expenses between involved parties at ‘events’. Some applications made by competitors that are already being used are ‘Groepie’ and ‘Wie betaalt wat?’. When the project is done, our application should be able to be used during the following example:

Three people go on a night out and use the application to make an ‘event’ and everyone is added to the ‘event’ on the application. They go out for ‘Dinner and drinks’ and take a cab home. Each friend paid for a different activity and added it to the application. The next day the friends can check their debts on the application for the activities and pay accordingly.


\section{Target or ambition level}
As discussed by the whole team, the primary goal is not solely focused on achieving a specific end grade. Instead, the focus lies on maximising their learning and delivering the best possible project. The team believes that prioritising these objectives will lead to a more than satisfactory grade.

To clarify the group’s focus, the rubric has been divided and the top priorities have been identified. ‘Product’ is the top priority as the team wants to treat the project as a trial run of a real development scenario and deliver a project that meets the standards. For the product itself, the main focus is not on making a product that has all the features and looks flashy, but a solid product which stands firmly and is user-friendly.
When the core product is solid, the team can devote time to further extensions and enhancements.

Next on the priority list is ‘Process’, as the members are learning new ways to develop a product, the tools that go along with developing it and the multiple non-technical skills every member will encounter. The team wants to make sure the process does not take a backseat, as the chances of using these skills in the future are very high, especially in teamwork environments, and see the benefits of acquiring this type of knowledge.


\section{Planning}
A significant part of teamwork is that the ‘Plan of Approach’ is transparent to everyone, including deadlines and meeting schedules. The members of Group 25 have a mandatory meeting with a TA, which from now on will be referred to as “the TA meeting”, every Tuesday from 13:45 to 14:30. During this meeting, the plans for the coming week will be discussed and made, including everyone’s tasks and deadlines. As stated in the ‘Teamwork’ assignment, the chair will divide the tasks, in consultation with the team.

In addition to the TA meeting, the members of Group 25 have also collectively agreed to hold a meeting on Friday afternoons during the self-study period, termed the “Friday meeting”. This meeting is a bit more casual and not mandatory (however, it is very much appreciated if all team members are present), and helps with creating more focus and motivation. The meeting does not have the formal structure as the TA meeting and can be more seen as a get-together and a good moment to discuss likely frictions and more personal level topics. More project-focused, members can provide updates on their progress, address any unforeseen issues or unresolved bugs, and discuss potential solutions together during the Friday meeting.

Group 25 has decided that the Product Owner has the final say in the final deliverable and will submit it to Brightspace on behalf of the entire project team. For further information about decision-making, go to section ‘8. Decision-making’.


\section{Behaviour}
The behaviour of every individual member of the team is very important for the whole team dynamic. The ‘1. Shared Team Values’ section describes a clear representation of how the team members should treat each other and how to maintain a peaceful and professional workplace environment.

\subsection{Disagreements}
In the event of a disagreement, it should be thoroughly discussed for a reasonable amount of time to strive for consensus. However, a majority vote will be conducted if consensus cannot be achieved. In case of a tie (i.e. a 3-3 vote), the tie-breaker will be the vote of the Product Owner.

\subsection{Being late}
Although being late is not very much appreciated, it is not the same in all cases. If a team member is late for the Friday meeting, it is not a big deal since it is not a mandatory meeting.

For the TA meeting, there are two different cases of being late: with or without announcing it upfront.

\subsubsection{Announcing upfront}
When a member informs the team well in advance that they will not be able to make the meeting in time, the group can take this into account, and lateness is not considered a problem. However, for repeated lateness and/or without a valid reason, the guidelines outlined for lateness without announcing it upfront will be followed.

\subsubsection{Without announcing upfront}
When a member is late and does not let the team know well in advance, it undermines the trust within the team, fails to meet their responsibilities and thus lets the team down. The member in question should make amends to restore trust and mend any disruption caused. The method of making amends is open for discussion among the team members. This ensures that accountability is upheld and that the team addresses any instances of letting each other down.

\subsection{Escalation}
Generally, the team resorts to involving external parties or seeks their mediation, when a conflict reaches a point where it can no longer be mediated by members of the team and/or the conflict delays the ongoing development of the product.


\section{Communication}
Group 25 operates through multiple communication channels, including WhatsApp, Discord and Mattermost. WhatsApp is used for general use, announcements, and Q\&A. These group chats are the main source of communication, apart from real-life communication. Discord is used for online meetings and Mattermost is only used for communication with the TA. Furthermore, frequent meetings are organised to decide on important issues and help each other if it is required.

When discussing the timing of communicating with team members, the team agreed on the following regarding replying to messages and notifications.
\begin{itemize}
    \item Members are not expected to always be constantly available or solely focused on the project at all times.
    \item Timely responses are only expected when there has been a prearranged agreement regarding the urgency or importance of the communication.
\end{itemize}

When communicating with team members, the team agreed on the following regarding replying to messages and notifications. Members are not expected to always be constantly available or solely focused on the project at all times. Timely responses are only expected when there has been a prearranged agreement regarding the urgency or importance of the communication.


\section{Commitment}
The basic requirements (100+ lines of code, 3+ commits, 1+ features) are a good measurement of the quantity of work an individual does. It is expected that each member fulfils these requirements during the weeks the completion of the requirements is tracked. If they are not reached, that team member should be contacted and asked for an explanation by the team. However, the team will inform the TA if the member in question can not be reached for multiple days.

While the team will appreciate all additional contributions, it is important to recognise that if a team member undertakes a significant part of the project independently, it limits the opportunity for others to acquire knowledge and experience in that area. Thus sticking to the tasks that are given to you by the chair, as outlined in the ‘4. Planning’ section, is crucial to ensure everyone has an equal distribution of workload and opportunities for learning.

To objectively assess the individual contribution of members, we use the built-in features in GitLab. The team works with Milestones and assigned issues. In our weekly check-ups on Friday as mentioned in “4. Planning” we assess if the work that was divided is on track and if all members have done their basic requirements and assigned issues.

Moreover, a team member should always try their utmost best to participate in all team meetings, the mandatory TA meetings and the Friday meetings, if possible and needed. Otherwise, after multiple skipped meetings the TA will be informed that a member is not actively participating in the team project. This will presumably have an effect on your mark or enrollment in the Object Orientated Programming Project.

The commitment of the chairs and minute-takers can be measured by reviewing the agenda sent by the chair and the updated agenda by the minute-taker. The team agrees it will be pretty clear if a chair or minute-taker is not committed enough. 


\section{Decision-making}
The first step of decision-making is thoroughly discussing the topic with the entire team such that everyone is well-informed. Decisions will be decided by consensus. However, since we have a limited amount of team meetings, if a consensus is not reached, the decision will be made by a majority vote. Since the topic has been discussed in depth, this enables them to make well-thought-out votes. However, if a person does not feel like they can make such a vote, a withdrawal from the vote is encouraged.

For the Product Owner role (technical matters), Rens Pols is the person appointed by the team responsible for the final decision if the vote comes to a tie.
For the Scrum Master role (other matters, such as planning and social matters), Sem van der Weijden is appointed by the team.


\section{Dealing with conflicts}
The process of dealing with conflicts within the team depends on the level of the conflict. If it is only a small disagreement, it is expected to be calmly and reasonably solved between the people involved. At the same time, members should pay attention to the reactions of other members of the group. If a member misunderstands, it's better to explain it to them face to face rather than it having a bigger impact weeks later. If the conflict escalates, involving other team members may be necessary. However, as stated in the section ‘5. Behaviour’, the team resorts to involving external parties or seeks their mediation, when a conflict reaches a point where it can no longer be mediated by members of the team and/or the conflict delays the ongoing development of the product.


\section{Consequences}
The consequences of specific situations have already been partly discussed in sections ‘5. Behaviour’ and ‘11. Commitment’.

The team has decided on the main topics that should have consequences:
\begin{enumerate}
    \item Not coming to mandatory meetings or being too late without notifying the group
    \item Code that doesn’t fulfil the basic quality requirements (rushed code)
    \item Communication that is intended to hurt other group members
    \item Not committing to assigned deadlines without notifying others
    \item Purposefully harming the project
\end{enumerate}

As for consequences to go with these topics, it is hard to decide predetermined actions. Things can happen for a multitude of reasons and in several severities, the group will come together to find a fitting consequence depending on the situation. Consequences could be:

\begin{itemize}
    \item Doing certain tasks for the project (testing/writing code/helping others)
    \item Informing the TA of the situation
    \item Bringing snacks to the next meeting
\end{itemize}

\section{Outside Collaboration}
Members of Group 25 have agreed that collaboration outside of mandatory meetings should maintain a relaxed, casual, yet professional work atmosphere, similar to that during the meetings. While the location for collaboration is not fixed, it will often occur in Echo or Pulse.

As previously outlined in the ‘Planning’ section, Group 25 has decided to hold the Friday meeting alongside the TA meeting, which will be a bit more casual and optional. While attendance is not mandatory, it is highly encouraged as this is also a time for team bonding and ensures everyone remains informed of the ongoing events. During this meeting, all members provide updates on their progress, address any unforeseen issues or unresolved bugs, and discuss potential solutions together.


\section{Scrum}
Group 25 uses the Scrum methodology. Below is a detailed explanation of what Scrum specifically is and how it is utilised.

\subsection{Introduction to Scrum}
The Scrum method is a popular framework for implementing agile project management. It is designed to help teams work together to efficiently address complex adaptive problems, while productively and creatively delivering products of the highest possible value. It aims to deliver new software capability every 2 - 4 weeks. It's characterised by a flexible, holistic product development strategy, where a development team works as a unit to reach a common goal. Scrum is widely used in software development but has been adapted for various other fields.

\subsection{Core Principles of Scrum}
\begin{itemize}
    \item \textbf{Empirical Process Control}: Scrum is based on empirical process control, or empiricism, which emphasises decision-making based on observation and experimentation rather than on detailed upfront planning.
    \item \textbf{Self-organisation}: Scrum teams are self-organising, meaning they decide internally how to best accomplish their work, rather than being directed by others outside the team.
    \item \textbf{Collaboration}: Emphasis is placed on collaboration within the team and with the stakeholders.
    \item \textbf{Value-based prioritisation}: Work is prioritised based on the value it will deliver to the customer.
\end{itemize}

\subsection{Scrum Framework Components}
\subsubsection{Roles}
\begin{itemize}
    \item \textbf{Product Owner}: Responsible for maximising the value of the product resulting from the work of the Development Team.
    \item \textbf{Scrum Master}: Acts as a coach to the team, helping members use the Scrum process to perform at the highest level.
    \item \textbf{Development Team}: Group members who Participate in the weekly work.
\end{itemize}

\subsubsection{Events (Ceremonies)}
\begin{itemize}
    \item \textbf{Sprint}: A time-boxed period (usually 2-4 weeks, in our project, it is 1 week) during which a "Done", usable, and potentially releasable product increment is created.
    \item \textbf{Sprint Planning}: A meeting at the beginning of each Sprint where the team decides what to accomplish during the Sprint.
    \item \textbf{Daily Scrum (Stand-up)}: Usually, it is a 15-minute time-boxed event for the Development Team to synchronise activities and create a plan for the next 24 hours. In our project, it is included in our weekly official meetings.
    \item \textbf{Sprint Review}: Held at the end of the Sprint to inspect the Increment and adapt the Product Backlog if needed, which is also a part of our weekly official meetings.
    \item \textbf{Sprint Retrospective}: A meeting after the Sprint Review to reflect on the past Sprint and plan for improvements to be implemented in the next Sprint, which is also a part of our weekly official meetings.
\end{itemize}

\subsubsection{Artefacts}
\begin{itemize}
    \item \textbf{Product Backlog}: An ordered list of everything that is known to be needed in the product, constantly evolving and managed by the Product Owner.
    \item \textbf{Sprint Backlog}: A set of items selected from the Product Backlog to be completed during the Sprint, plus a plan for delivering the product Increment and realising the Sprint Goal.
    \item \textbf{Increment}: The sum of all the Product Backlog items completed during a Sprint and all previous Sprints.
\end{itemize}


\section{Summary}
Group 25 upholds a set of core values, including respect, integrity, openness, inclusivity, casual professionalism, open communication, team effort, commitment and resource sharing. The assignment involves developing an "expense calculator" application. Rather than fixating only on achieving specific grades, the team prioritises learning and delivering a high-quality project. The team will carry out transparent planning with mandatory and optional meetings. As for behaviour norms, communication, commitment and decision-making, the members have reached a consensus on them and come up with some solutions that all agree on. Outside collaboration is actively encouraged as this among other things strengthens the bond of the team. Moreover, the team embraces the Scrum methodology, which will further enhance the team's ability to adapt and deliver results iteratively, promoting efficiency and flexibility in the team's development process.


\end{document}
\endinput
